\documentclass[pdftex, a4paper, 11pt]{article}
\usepackage[utf8]{inputenc}
\usepackage[portuguese]{babel}
\usepackage{fancyhdr,graphicx}
\usepackage[hmargin=2cm,vmargin=2cm,a4paper]{geometry}
\usepackage{hyperref}

\pagestyle{fancy}
\lhead{\scshape Curriculum Vitae}
\rhead{\itshape Robson Cardoso dos Santos}
\rfoot{\footnotesize pag. \thepage}
\cfoot{}
\lfoot{{\footnotesize Atualizado em:} \today}
\renewcommand{\headrulewidth}{1pt}
\renewcommand{\footrulewidth}{1pt}

\begin{document}
\vspace*{.3cm}
\begin{center}
  \rule{.8\textwidth}{1pt}\\[10pt]
  \begin{minipage}{.55\textwidth}
    \LARGE\textbf{Robson Cardoso dos Santos}\\[13pt]
    \small Rio Xingú, 1716\\
    82840-300 - Bairro Alto\\
    Curitiba - PR\\[6pt]
    \textbf{e-mail: cardoso.rcs@gmail.com}\\
%%    \small \url{http://www.ecarlesso.org}\\
    Telefone: 41 99944-2409\\[6pt]
    \small Nacionalidade: Brasileira\\
    \small Data de nascimento: 09/09/1979\\
%%    \small Estado civil: Solteiro\\
  \end{minipage}
  \begin{minipage}{.2\textwidth}
    \includegraphics[width=\textwidth]{dummy.png}
  \end{minipage}\\[5pt]
  \rule{.8\textwidth}{1pt}
\end{center}
\vspace*{1cm}

%% \section*{Posição Atual:}

%% \hspace{0.58cm}Aluno do terceiro ano de Ciência da Computação da Universidade Federal do Paraná.\\

\section*{Formação Acadêmica}
\begin{tabular}{ll}
  \textbf{2009 - atual} & Graduando de Bacharelado em Ciência da Computação.\\
  & Universidade Federal do Paraná, UFPR, Curitiba\\
  & \\
%%  \textbf{2009} & Graduação imcompleta em Licenciatura em Física.\\
%%  & Universidade Federal do Paraná, UFPR, Curitiba\\
%%  & \\
  \textbf{1995 - 2000} & Ensino Profissional de nível técnico em Mecânica.\\
  & Centro Federal de Educação Tecnológica do Paraná, CEFET, Curitiba\\
\end{tabular}

\section*{Experiência}
\begin{itemize}
\item Desenvolvimento e manutenção de aplicativos em COBOL/CICS utilizando TSO e a ferramenta SDF II para mapas CICS;
\item Desenvolvimento do sistema gerenciador dos Telecentros para o Ministério das Comunicações, para plataforma Linux, utilizando Python, Django e Shell Script, através do C3SL/UFPR;
\item Participação no desenvolvimento/adaptação do Linux Educacional para o Ministério das Comunicações, com auxílio de ferramentas de controle de versão e Shell Script para automatização de tarefas de empacotamento e commit de versionamento, através do C3SL/UFPR;
\item Gerenciamento de sistemas Linux, para configuração de firewall e reencaminhamento de conexões, com ferramentas nativas do sistema Linux;
\item Criação de aplicações web MVC, usando Python e Django;
\item Empacotamento Debian;
\item Desenvolvimento de um Sistema de Gerenciamento de Conteúdo - CMS - em Shell Script.
\item Manutenção e criação de softwares para aquisição e controle de sinais digitais e analógicos usando LabView; %%, para execução de ensaios destrutivos e não-destrutivos em componentes mecânicos;
\item Projetos e desenhos mecânicos, ensaios mecânicos não destrutivos e destrutivos;
\item Usinagem, soldagem e ajustagem mecânica.
\end{itemize}

\section*{Conhecimentos de Informática}
\begin{description}
\item[S.O.:] GNU/Linux(ArchLinux, Debian, RedHat), Mac OS, Windows.
\item[Programação:] Conhecimento em C, Python, Django, javascript, COBOL e VHDL. Domínio em desenvolvimento com LabView e Shell Scripts.
\item[Outros:] \LaTeX, emacs, sistema de controle de versão(GIT), SQL, HTML, CSS, Gimp, Inkscape, Arduino, Autodesk Inventor.
\end{description}

\section*{Atuação Profissional}
\begin{itemize}
\item \textbf{TIVIT}\\
  \begin{tabular}{lp{13cm}}
    \textbf{2015 - 2015} & Vínculo: Colaborador, Enquadramento funcional: Analista Desenvolvedor Júnior, 40 horas.\\
  \end{tabular}

\item \textbf{Stefanini}\\
  \begin{tabular}{lp{13cm}}
    \textbf{2014 - 2015} & Vínculo: Estagiário, Enquadramento funcional: Programador, 30 horas.\\
  \end{tabular}

\item \textbf{Centro de Computação Científica e Software Livre - C3SL}\\
  \begin{tabular}{lp{13cm}}
    \textbf{2011 - 2013} & Vínculo: Estagiário, Enquadramento funcional: Programador, 20 horas.\\
  \end{tabular}
  
\item \textbf{Instituto de Tecnologia para o Desenvolvimento - LACTEC}\\
  \begin{tabular}{lp{13cm}}
    \textbf{2001 - 2011} & Vínculo: Colaborador, Enquadramento funcional: Técnico Mecânico, 40 horas, dedicação exclusiva.\\
  \end{tabular}
\end{itemize}

%% \section*{Interesse pessoal}
%% \begin{itemize}
%% \item Informática, com particular atenção a programação;
%% \item Software livre;
%% \item Hardware, com interação software / eletrônica
%% \end{itemize}

\section*{Idiomas}
\begin{description}
\item[Inglês:] Bom nível de compreensão e escrita, fala razoável;
\end{description}


\vfill

%% Autorizzo il trattamento dei dati da me forniti ai sensi della legge
%% 675/96 sulla privacy.

\vspace{1cm}

%% \footnotesize {Questo curriculum \`e hostato sotto {\em git}. \`E possibile scaricare la versione aggiornata:}
%% \begin{verbatim}
%%     $ git clone git://repo.or.cz/ecarlesso_cv.git
%% \end{verbatim}
\end{document}
